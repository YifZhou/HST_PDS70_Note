\documentclass[modern]{aastex62}

\usepackage{stix}
\newcommand{\vdag}{(v)^\dagger}
\newcommand\aastex{AAS\TeX}
\newcommand\latex{La\TeX}

\graphicspath{{./}{figures/}}
\shorttitle{PDS70}
\shortauthors{Zhou et al.}
%%
%% You can add a light gray and diagonal water-mark to the first page 
%% with this command:
% \watermark{text}
%% where "text", e.g. DRAFT, is the text to appear.  If the text is 
%% long you can control the water-mark size with:
%  \setwatermarkfontsize{dimension}
%% where dimension is any recognized LaTeX dimension, e.g. pt, in, etc.
%%
%%%%%%%%%%%%%%%%%%%%%%%%%%%%%%%%%%%%%%%%%%%%%%%%%%%%%%%%%%%%%%%%%%%%%%%%%%%%%%%%

%% This is the end of the preamble.  Indicate the beginning of the
%% manuscript itself with \begin{document}.

\begin{document}

\title{PDS70 Notes}

\correspondingauthor{Yifan Zhou}
\email{yzhou@as.arizona.edu}

\author{Yifan Zhou}
\affil{Steward Observatory}
\begin{abstract}
  
\end{abstract}
\keywords{}
\section{Information from literatures}
Results from \citet{Keppler2018,Wagner2018} are in Table~\ref{tab:PDS70}
\begin{deluxetable}{ll}
  \tablenum{1}
  \tablecaption{Planetary-mass Information Summary \label{tab:PDS70}}
  \tablehead{
    \colhead{Name} &
    \colhead{Value}
  }
  \startdata
  J Phot & $9.55\pm0.024$ mag\\
  Distance & $113.43\pm0.52$ pc\\
  Separation & 0.195 mas\\
  PA & $155^{\circ}$\\
  H-$\alpha$ contrast & $10^{-3}$\\
  \enddata
\end{deluxetable}

\section{TinyTim PSF fitting results}

TinyTim PSFs were fit to two HST/UVIS \texttt{flt} frames. The aim was to measure the centroid position and brightness of PDS70A. The residuals for the best-fit subtraction results are shown in Figure \ref{fig:tinytim}. The best-fit parameters are in Table \ref{tab:tinytim}

\begin{figure*}[htbp]
  \centering
  \plottwo{figures/flt_1_TinyTim_residual.pdf}{figures/flt_2_TinyTim_residual.pdf}
  \caption{TinyTim PSF subtraction residuals}
  \label{fig:tinytim}
\end{figure*}

% table for the tinytim fit result

\begin{deluxetable}{cccccc}
  \tablenum{2}
  \tablecaption{TinyTim fitting results}\label{tab:tinytim}
  \tablehead{
    \colhead{XC1}&
    \colhead{YC1}&
    \colhead{flux1}&
    \colhead{XC2}&
    \colhead{YC2}&
    \colhead{flux2}\\
    \colhead{pixel}&
    \colhead{pixel}&
    \colhead{e$^{-}$/s}&
    \colhead{pixel}&
    \colhead{pixel}&
    \colhead{e$^{-}$/s}
  }
  \startdata
    454.60&
    489.40&
    715848.9&
    457.23&
    491.90&
    687965.953    
    \enddata
  \end{deluxetable}
  
%%% Local Variables:
%%% mode: latex
%%% TeX-master: "main"
%%% End:


\section{KLIP reduction}
Next I reduced the data using KLIP method \citep{Soummer2012}. The PSF library was the recently released WFC3/UVIS PSF library. I used the entire F656N library without any further PSF selection (assuming all the PSFs are good).


\section{PSF injection test}
I used PSF injection test to examine the contrast limit of the observations.

\bibliographystyle{yahapj}
\bibliography{library}

\end{document}

% End of file `sample62.tex'.

%%% Local Variables:
%%% mode: latex
%%% TeX-master: t
%%% End:
